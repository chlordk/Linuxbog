\documentclass[a4paper]{article}
 \usepackage[danish]{babel}
 \usepackage[T1]{fontenc}
 
 \title{Strukturovervejelser for \\
   ``Linux - Friheden til at programmere''}
 
 \author{Jacob Sparre Andersen}

\begin{document}
 \maketitle
 
 \section{Kapitlet ``Vælg sprog''}

 Alfabetisk ordnede kapitler: Ada, C, C++, Cobol, Forth, \ldots, Zsh.
 
 For hvert sprog skal der være en beskrivelse af sprogets art, fordele
 og ulemper.

 \section{Kapitlet ``Vælg værktøjer''}

 \begin{enumerate}
  \item{\bf Versionsstyring}: CVS, diff, patch, Bugzilla, \ldots
  \item{\bf Redigering}: vi, jed, pico, \ldots
  \item{\bf GUI-byggere}: GLADE(?)
  \item{\bf Oversættelsesstyring}; make, autoconf, automake.
  \item{\bf Prettyprintere}: GRASP, a2ps, \ldots
  \item{\bf Integrerede udviklingsmiljøer}: Anjuta, GPS, KDevelop,
    Emacs, \ldots
  \item{\bf Fejlsøgning}: GVD, gdb, efence, \ldots
 \end{enumerate}
 
 For hvert afsnit i kapitlet gives en kort beskrivelse af formålet med
 den klasse værktøjer.

 \section{Kapitlet ``Kom i gang med \ldots''}
 
 Alle sprog og værktøjer i alfabetisk orden.  For hvert af dem gives
 der en kort brugsanvisning og henvisninger til andet materiale
 (lærebøger, kurser og dokumentation).

 \section{Appendikset ``Installationsvejledninger''}
 
 Da vi går ud fra at alt er på plads på et ordentligt system, er
 installationsvejledningerne placeret i et appendiks.

 \section{Datastruktur til automatisk produktion}

 \begin{itemize}
  \item{\bf navn}: ``Ada'' | ``CVS'' | ``diff'' | ``Fortran'' | \ldots
  \item{\bf type}: ``sprog'' | ``versionsstyring'' | ``redigering'' |
    \ldots
  \item{\bf beskrivelse}: XML Docbook-kode svarende til indeni et
    ``sect1''-element.
  \item{\bf introduktion}: XML Docbook-kode svarende til indeni et
    ``sect1''-element.
  \item{\bf henvisninger}: (format?)
  \item{\bf installationsvejledning}: XML Docbook-kode svarende til
    indeni et ``sect1''-element.
  \item{\bf korrekturlæsning}: Datoer for sidste korrekturlæsninger af
    beskrivelse, introduktion, henvisninger og
    installationsvejledning.
  \item{\bf udløbsdato}: Datoer for hvornår henholdsvis beskrivelse,
    introduktion, henvisninger og installationsvejledning senest skal
    opdateres.
 \end{itemize}
 
 Et sprog eller et værktøj kommer først med i bogen, når alle punkter
 på ovenstående liste er udfyldte. Hvis et der ikke er læst korrektur
 på et afsnit siden sidst det blev rettet kommer det ikke med.
 Tilsvarende kommer et afsnit heller ikke med efter sin udløbsdato.

\end{document}