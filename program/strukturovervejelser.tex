\documentclass[a4paper]{article}
 \usepackage[danish]{babel}
 \usepackage[T1]{fontenc}
 
 \title{Strukturovervejelser for \\
   ``Linux - Friheden til at programmere''}
 
 \author{Jacob Sparre Andersen}

\begin{document}
 \maketitle
 
 \section{Kapitlet ``V�lg sprog''}

 Alfabetisk ordnede kapitler: Ada, C, C++, Cobol, Forth, \ldots, Zsh.
 
 For hvert sprog skal der v�re en beskrivelse af sprogets art, fordele
 og ulemper.

 \section{Kapitlet ``V�lg v�rkt�jer''}

 \begin{enumerate}
  \item{\bf Versionsstyring}: CVS, diff, patch, Bugzilla, \ldots
  \item{\bf Redigering}: vi, jed, pico, \ldots
  \item{\bf GUI-byggere}: GLADE(?)
  \item{\bf Overs�ttelsesstyring}; make, autoconf, automake.
  \item{\bf Prettyprintere}: GRASP, a2ps, \ldots
  \item{\bf Integrerede udviklingsmilj�er}: Anjuta, GPS, KDevelop,
    Emacs, \ldots
  \item{\bf Fejls�gning}: GVD, gdb, efence, \ldots
 \end{enumerate}
 
 For hvert afsnit i kapitlet gives en kort beskrivelse af form�let med
 den klasse v�rkt�jer.

 \section{Kapitlet ``Kom i gang med \ldots''}
 
 Alle sprog og v�rkt�jer i alfabetisk orden.  For hvert af dem gives
 der en kort brugsanvisning og henvisninger til andet materiale
 (l�reb�ger, kurser og dokumentation).

 \section{Appendikset ``Installationsvejledninger''}
 
 Da vi g�r ud fra at alt er p� plads p� et ordentligt system, er
 installationsvejledningerne placeret i et appendiks.

 \section{Datastruktur til automatisk produktion}

 \begin{itemize}
  \item{\bf navn}: ``Ada'' | ``CVS'' | ``diff'' | ``Fortran'' | \ldots
  \item{\bf type}: ``sprog'' | ``versionsstyring'' | ``redigering'' |
    \ldots
  \item{\bf beskrivelse}: XML Docbook-kode svarende til indeni et
    ``sect1''-element.
  \item{\bf introduktion}: XML Docbook-kode svarende til indeni et
    ``sect1''-element.
  \item{\bf henvisninger}: (format?)
  \item{\bf installationsvejledning}: XML Docbook-kode svarende til
    indeni et ``sect1''-element.
  \item{\bf korrekturl�sning}: Datoer for sidste korrekturl�sninger af
    beskrivelse, introduktion, henvisninger og
    installationsvejledning.
  \item{\bf udl�bsdato}: Datoer for hvorn�r henholdsvis beskrivelse,
    introduktion, henvisninger og installationsvejledning senest skal
    opdateres.
 \end{itemize}
 
 Et sprog eller et v�rkt�j kommer f�rst med i bogen, n�r alle punkter
 p� ovenst�ende liste er udfyldte. Hvis et der ikke er l�st korrektur
 p� et afsnit siden sidst det blev rettet kommer det ikke med.
 Tilsvarende kommer et afsnit heller ikke med efter sin udl�bsdato.

\end{document}